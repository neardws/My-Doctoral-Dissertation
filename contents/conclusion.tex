\chapter{总结与展望}\label{section 7}

车联网是通过网络技术将车辆、道路基础设施以及其他联网设备连接起来,实现车辆信息的互联互通及共享的新兴技术。
车联网有助于提高行车安全、促进交通流畅、降低能源消耗、提升驾驶体验和推动智能制造等方面,对于我国现代化城市建设和汽车产业发展都有着重要的推动作用。
而车载信息物理融合系统已成为支撑车联网中各类智慧交通系统应用的关键。
本文致力于从服务架构、评估指标设计、异构资源协同优化、质量-开销均衡和原型系统实现五个方面协同驱动面向异构车联网的车载信息物理融合系统。
首先,面向高动态异构车联网,融合不同的计算范式与服务架构是实现车载信息物理融合的基础。
其次,面向分布式时变物理环境,有效的数据获取与建模评估是驱动车载信息物理融合的核心。
再次,面对动态异构节点资源,高效的任务调度与资源分配是进一步提升智慧交通系统服务质量的关键。
另外,面向多元智慧交通系统应用需求,满足差异性的系统质量与系统开销需求是驱动车载信息物理融合的另一关键。
最后,面向复杂的真实车联网环境,基于车载信息物理融合系统进行有效设计并实现具体系统原型是具有挑战的。
本文的主要贡献如下:

\circled{1} 基于软件定义网络和边缘计算的异构车联网架构研究。
首先,将SND与边缘计算融入车联网服务架构,旨在最大化SDN逻辑上集中控制与分布式边缘计算服务的协同效应。
具体地,设计了分层车联网架构,其中包括应用层、控制层、虚拟层以及数据层,通过SDN思想将控制平面与数据平面分离实现逻辑上的集中控制,利用分布式边缘计算服务的异构资源进一步增强系统性能。
其次,针对所提服务架构分析了其中的挑战与机遇,并进一步提出跨层协议栈对未来研究方向进行讨论。
最后,通过两个典型应用的案例研究,验证了所提服务架构的巨大潜力以及对未来智慧交通系统发展提供启示。

\circled{2} 面向车载边缘计算的VCPS评估指标(Age of View)设计与优化策略研究。
首先,提出了面向车载边缘计算的协同感知与异质信息融合框架,其中边缘节点通过融合车辆感知信息构建边缘视图以反映实时车联网环境。
其次,建立了基于多类M/G/1优先队列的感知信息排队模型,并进一步基于异质信息的时效性、完整性和一致性建模设计了崭新的车载信息物理融合评估指标(Age of View),并形式化定义了视图质量最大化问题。
再次,提出了基于差分奖励的多智能体强化学习(MDR)算法,其中车辆动作空间包括信息感知频率与上传优先级,边缘节点基于车辆预测轨迹和视图需求分配V2I带宽,并通过基于差分奖励的信用分配机制评估车辆对于视图构建的贡献。
最后,仿真实验结果表明,MDR算法能有效提高车载信息物理融合系统的质量。

\circled{3} 面向NOMA车载边缘计算的异构资源协同优化策略研究。
首先,提出了基于NOMA的车载边缘计算框架,其中边缘节点协同分配异构资源以对计算任务进行实时处理。
其次,考虑了NOMA车联网中域内和域间干扰,建立了V2I传输模型,并形式化定义了最大化任务服务率的协同资源优化问题。
再次,将协同资源优化问题分解为任务卸载与资源分配子问题,其中将任务卸载子问题建模为严格势博弈模型,并进一步提出多智能体分布式深度确定性梯度策略(MAD4PG)算法来实现纳什均衡,另一方面,将资源分配子问题分解为两个独立凸优化问题,并分别使用基于梯度的迭代方法和KKT条件得到最优解。
最后,仿真实验结果表明,MAD4PG算法能最大化异构资源利用率,能有效提高任务完成率。

\circled{4} 面向车载信息物理融合系统的质量/开销均衡优化策略研究。
首先,提出了车载信息物理融合框架,其中边缘节点基于车辆感知信息对车联网中物理实体建立相应的数字孪生模型。
其次,考虑数字孪生的时效性、一致性和冗余度,以及感知与传输开销,建立了车载信息物理融合系统质量和开销模型,并进一步形式化定义了最大化VCPS质量且最小化VCPS开销的双目标优化问题。
再次,提出了多目标多智能体强化学习(MAMO)算法,其中提出了决斗评论家网络,基于状态价值和动作优势来评估智能体的动作。
最后,仿真实验结果表明,MAMO算法可以实现VCPS质量和开销均衡以满足不同的智慧交通系统应用需求。

\circled{5} 基于车载信息物理融合的超视距碰撞预警原型系统设计与实现。
首先,提出了超视距碰撞预警场景,其中车辆由于非视距(NLoS)问题存在潜在碰撞风险,提出基于V2X通信的碰撞预警解决方案。
其次,提出基于视图修正的碰撞预警(VCCW)算法,具体地,基于现场测试获得的V2I应用层传输时延数据,建立了基于稳定分布的V2I时延拟合模型。
另一方面,设计了基于数据上传频率和车辆状态历史信息的无线传输丢包检测机制。
通过时延估计和丢包检测对视图进行校准,以提供更加准确实时的碰撞预警服务。
再次,仿真实验结果表明,VCCW能有效提高碰撞预警的查准率和查全率。
最后,搭建了基于C-V2X的硬件在环试验平台,对C-V2X端到端时延进行了分析,进一步基于无人小车搭建了验证平台,部署了VCCW算法并在真实复杂车联网环境中实现了超视距碰撞预警原型系统,验证了VCCW算法的有效性。

本文主要针对面向异构车联网的车载信息物理融合系统关键技术开展了研究,并取得了一定的成果。
但作为探索VCPS的早期阶段,本文工作无法完全解决所有挑战,有待进一步探索和解决。
在未来工作中,本文将进一步研究边缘节点之间的合作,以扩大支持的ITS应用,并提高整体系统的性能。
其次,通过考虑车辆移动性和边缘节点之间合作计算的内在关系来进一步提高系统性能。
此外,将考虑车联网端边云架构,通过利用车辆、边缘节点和云协同来提高性能。
